Accidental Complexity in contemporary software development is high, especially in UI development. A big chunk of UI development time is not spent on delivering business value but infrastructure work. This thesis explores an efficient approach to UI development by reducing Accidental Complexity.

We analyze hypermedia specifications and develop a proof of concept for a UI framework by implementing two use cases. The first use case is a scenario in home automation and it is concerned with non-interactive UIs. The second use case describes a project management tool with focus on user interaction.

The combination of linked data, web components and HATEOAS reduces the complexity in UI and client development. Using the developed UI framework reduces cognitive load, eliminates Complexity caused by State and Complexity caused by Control and significantly drives down maintenance costs.

The approach to UI development we describe constitutes an improvement to contemporary ways.
