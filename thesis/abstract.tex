Accidental complexity in contemporary software development is high, especially so in UI development. A big chunk of UI development time is not spent on solving problems in the business domain. This thesis explores a new approach to UI development by reducing the accidental complexity. I develop a UI framework that enables developers to create UIs that are driven by linked data. In combination with concepts of task-based computing, this allows to decouple the client containing the UI from the backend service. The UI produced by the UI framework can consume every API that conforms to a specification because it doesn't contain any business knowledge.

By making heavy use of linked-data, the UI components can be re-used across various contexts like organization boundaries. The choice of JSON-LD as linked-data serialization format makes sure that the UI framework plays nicely with existing tooling.

A proof-of-concept was built by solving two real-world use cases: A home automation use cases demonstrates the rendering capabilities of the UI framework. This was achieved by providing an API to developers to plug-in custom renderers. The second use case showcases the interaction handling of the UI framework.
