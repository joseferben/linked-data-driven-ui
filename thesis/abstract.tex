\textit{Accidental Complexity} in contemporary software development is high, especially in development of user interfaces. A big chunk of user interface development time is not spent on delivering business value but infrastructure work. This thesis explores an efficient approach to development of user interfaces by reducing \textit{Accidental Complexity}.

We analyze hypermedia specifications and develop a proof of concept for a user interface framework by implementing two use cases. The first use case is a scenario in home automation and it is concerned with non-interactive user interfaces. The second use case describes a project management tool with focus on user interaction.

The combination of linked data, web components and \textit{Hypermedia As The Engine Of Application State} reduces the complexity in development of user interfaces. Usage of the implemented framework reduces \gls{cognitive load}, decreases \textit{Complexity caused by State} and \textit{Complexity caused by Control} and significantly drives down maintenance costs.

The suggested approach to development of user interfaces poses an improvement over approaches without linked data and hypermedia.
