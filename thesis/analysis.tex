\section{Analysis}
\subsection{Risk analysis}\label{sec:riskanalysis}

\paragraph{Methodology}
We identify risks that could affect the implementation of the UI framework and the use cases. Each risk has a probability of occurrence as shown in table \ref{tab:probability} and a severity as explained in table \ref{tab:severity}. Multiplication of both scores yields the risk factor. We list all risks sorted descending by their risk factor in table \ref{tab:riskanalysis}. A risk factor greater or equal \textbf{8} indicates a risk that is likely to cause substantial project damage. \\
Furthermore, decisions that are made in section \ref{sec:proofofconcept} are based on this sorted list of risks.

\begin{table}[!htb]
  \begin{center}
    \begin{tabular}{|l|l|l|}
      \hline
      \textbf{Probability} & \textbf{Score} & \textbf{Description} \\
      \hline
      0 <= p <= 0.25 & 1 & The probability of occurrence of this risk is minimal. \\
      \hline
      0.25 < p <= 0.5 & 2 & The risk might occur with a low probability. \\
      \hline
      0.5 < p <= 0.75 & 3 & There is a significant chance for the risk to occur. \\
      \hline
      0.75 < p <= 1 & 4 & The probability of occurrence of this risk is high. \\
      \hline
    \end{tabular}
    \caption{The probability of occurrence is mapped to a score which is used in the risk analysis.}
    \label{tab:probability}
  \end{center}
\end{table}

\begin{table}[!htb]
  \begin{center}
    \begin{tabular}{|l|l|l|}
      \hline
      \textbf{Severity} & \textbf{Score} & \textbf{Project Damage} \\
      \hline
      insignificant & 1 & \begin{tabular}[x]{@{}l@{}}It is possible to implement the UI framework and the use cases \\ but the results might be poor.\end{tabular} \\
      \hline
      moderate & 2 & \begin{tabular}[x]{@{}l@{}}It is possible to implement the UI framework and the use cases \\ but the results might be unusable.\end{tabular} \\
      \hline
      high & 3 & \begin{tabular}[x]{@{}l@{}}It is possible to implement the UI framework and the use cases. \\ That approach to UI development is less efficient and more \\ costly than UI development without the UI framework.\end{tabular} \\
      \hline
      fatal & 4 & \begin{tabular}[x]{@{}l@{}}It is not possible to implement the UI framework and the use cases.\end{tabular} \\
      \hline
    \end{tabular}
    \caption{The level of severity is mapped to a score which is used in the risk analysis.}
    \label{tab:severity}
  \end{center}
\end{table}

\begin{table}[!htb]
  \begin{center}
    \begin{tabular}{|l|l|}
      \hline
      \textbf{ID} & R1 \\
      \hline
      \textbf{Title} & Incomplete HATEOAS specification \\
      \hline
      \textbf{Probability of occurrence} & 2 \\
      \hline
      \textbf{Severity} & 4 \\
      \hline
      \textbf{Risk Factor} & 8 \\
      \hline
      \textbf{Description} & \begin{tabular}[x]{@{}l@{}}The HATEOAS specification in use is either an early draft or it \\ has been abandoned. In the worst case, the specification superficially \\ appears to be complete while not providing crucial features that \\ are needed by the use cases. Parts of it have to be written during the \\ thesis work.\end{tabular} \\
      \hline
    \end{tabular}
    \caption{Source of risk R1.}
  \end{center}
\end{table}

\begin{table}[!htb]
  \begin{center}
    \begin{tabular}{|l|l|}
      \hline
      \textbf{ID} & R2 \\
      \hline
      \textbf{Title} & Unstable HATEOAS specification and tooling\\
      \hline
      \textbf{Probability of occurrence} & 3 \\
      \hline
      \textbf{Severity} & 1 \\
      \hline
      \textbf{Risk Factor} & 3 \\
      \hline
      \textbf{Description} & \begin{tabular}[x]{@{}l@{}}The HATEOAS specification in use and reference implementations \\ are under development and parts are rewritten frequently. For the \\ thesis work a version has to be pinpointed.\end{tabular} \\
      \hline
    \end{tabular}
    \caption{Source of risk R2.}
  \end{center}
\end{table}

\clearpage

\begin{table}[!htb]
  \begin{center}
    \begin{tabular}{|l|l|}
      \hline
      \textbf{ID} & R3 \\
      \hline
      \textbf{Title} & Lack of HATEOAS tooling \\
      \hline
      \textbf{Probability of occurrence} & 2 \\
      \hline
      \textbf{Severity} & 2 \\
      \hline
      \textbf{Risk Factor} & 4 \\
      \hline
      \textbf{Description} & \begin{tabular}[x]{@{}l@{}}The HATEOAS specification in use has no reference implementation, \\ little working examples and no tools for generating and consuming \\ HATEOAS APIs. All tools have to be written from scratch.\end{tabular} \\
      \hline
    \end{tabular}
    \caption{Source of risk R3.}
  \end{center}
\end{table}

\begin{table}[!htb]
  \begin{center}
    \begin{tabular}{|l|l|}
      \hline
      \textbf{ID} & R4 \\
      \hline
      \textbf{Title} & Lack of JSON-LD tooling \\
      \hline
      \textbf{Probability of occurrence} & 1 \\
      \hline
      \textbf{Severity} & 3 \\
      \hline
      \textbf{Risk Factor} & 3 \\
      \hline
      \textbf{Description} & \begin{tabular}[x]{@{}l@{}}There are no tools that implement the JSON-LD operations \\ discussed in section \ref{jsonld}. The used languages and frameworks don't \\ support JSON as first-class citizen \footnote{A language treats JSON as first-class citizen if it supports it natively.}. All tools have to be written \\ from scratch.\end{tabular} \\
      \hline
    \end{tabular}
    \caption{Source of risk R4.}
  \end{center}
\end{table}

\begin{table}[!htb]
  \begin{center}
    \begin{tabular}{|l|l|}
      \hline
      \textbf{ID} & R5 \\
      \hline
      \textbf{Title} & Lack of design language implementations \\
      \hline
      \textbf{Probability of occurrence} & 1 \\
      \hline
      \textbf{Severity} & 4 \\
      \hline
      \textbf{Risk Factor} & 4 \\
      \hline
      \textbf{Description} & \begin{tabular}[x]{@{}l@{}}The UI framework provides usable UIs out of the box. These  \\ are created by using a design language implementation which \\ takes care of portability across browsers. The implementation provides \\ the authors with best practices in UI design and layout. In case \\ there is a lack of design language implementations, one has to \\ be written from scratch.\end{tabular} \\
      \hline
    \end{tabular}
    \caption{Source of risk R5.}
  \end{center}
\end{table}

\clearpage

\begin{table}[!htb]
  \begin{center}
    \begin{tabular}{|l|l|}
      \hline
      \textbf{ID} & R6 \\
      \hline
      \textbf{Title} & Linked data-driven UIs are slow \\
      \hline
      \textbf{Probability of occurrence} & 3 \\
      \hline
      \textbf{Severity} & 2 \\
      \hline
      \textbf{Risk Factor} & 6 \\
      \hline
      \textbf{Description} & \begin{tabular}[x]{@{}l@{}}JSON-LD operations as discussed in section \ref{jsonld} require multiple \\ requests per user request. This adds some overhead to \\ the client-server communication which can decrease performance. \\ If those additional requests can not be mitigated by caching, the \\ overhead might affect the \gls{ux}. \end{tabular} \\
      \hline
    \end{tabular}
    \caption{Source of risk R6.}
  \end{center}
\end{table}

\begin{table}[!htb]
  \begin{center}
    \begin{tabular}{|l|l|l|}
      \hline
      \textbf{ID} & \textbf{Risk Factor} & \textbf{Title} \\
      \hline
      \rowcolor{yellow}
      R1 & 8 & Incomplete HATEOAS specification \\
      \hline
      R6 & 6 & Linked data-driven UIs are slow \\
      \hline
      R5 & 4 & Lack of design language implementations \\
      \hline
      R3 & 4 & Lack of HATEOAS tooling \\
      \hline
      R2 & 3 & Unstable HATEOAS specification and tooling \\
      \hline
      R4 & 3 & Lack of JSON-LD tooling \\
      \hline
    \end{tabular}
    \caption{The result of the risk analysis shows the list of all risks sorted descending by their total risk factor. R1 has a large impact on the success of the project.}
    \label{tab:riskanalysis}
  \end{center}
\end{table}

\subsection{Conceptual analysis of user interfaces}
For a system to be useful, it has to interact with its environment. These interactions happen through multiple types of interfaces, which are used by either other systems or human users.
Interfaces for human users share fundamental properties. By observing their usage we can analyze the main characteristics. A typical scenario of UI usage is browsing a website. At all times, while using the website, the user either reads or clicks.

The UI has to present information that the user \textbf{reads} and it has to provide means for \textbf{interaction}. In terms of operations on data, this is equivalent to \textbf{querying} and \textbf{mutating or commanding}.

\begin{enumerate}
  \item Query: The user is able to retrieve information by looking at the UI
  \item Interaction: The user is able to interact with the UI
\end{enumerate}

These two fundamental properties of UIs are driving the definition and implementation of the two use cases.

\subsection{Analysis of \gls{hateoas} API specifications}\label{sec:analysishateoas}
The goal is to chose a specification among existing \gls{hateoas} API specifications. The real-world uses cases will implement HTTP APIs that conform to this specification \footnote{The actual implementation of those APIs is not the concern of this thesis. The focus is on the choice and evaluation of a HATEOAS specification.}. The client that consumes that API is merely a \gls{console}. This implies the client doesn't contain any knowledge about the API implementation. \\
We enumerate the requirements for a \gls{hateoas} specification from the point of view of a consumer.

\paragraph{Self documenting}
The API should provide its own documentation so the consumer is not reliant on additional tools to consume documentation. The API should document itself which requires a mechanism to attach meta data to the response payload.

\paragraph{Discoverability}
The consumer should be able to autonomously discover the data model of an API by autonomously building and invoking requests.

\paragraph{Actions and Operations}
The API should provide a description on how to interact with it. It publishes the information that is needed by the consumer to build and invoke requests.

\paragraph{Uses linked data}
Based on the discussion in section \ref{linkeddata} about the benefits of \gls{linkeddata}, we explore \gls{linkeddata} as the key component to efficient UI development. The combination with web component based rendering discussed in section \ref{webcomponents} allows UI developers to think in terms of components and data.

\subsubsection{Hydra}
It is remarkable that at the time of writing this thesis, only one \gls{hateoas} specification exists that satisfies the requirements stated in section \ref{sec:analysishateoas}.

\textit{``\href{http://www.hydra-cg.com/}{Hydra} is a lightweight specification to create hypermedia-driven Web APIs. By specifying a number of concepts commonly used in Web APIs, it enables the creation of generic API clients \citep{hydraspecs}.''} At the time of writing, the Hydra specification exists as an unofficial draft.

Two benefits of APIs supporting hypermedia are \textbf{discoverability} and \textbf{self documentation}.

\paragraph{Discoverability} Discoverability is given by the RESTful nature of any hypermedia API. The API provides an entry point and each resource describes the means to fetch other resources using \gls{uri}s. The client is able to traverse the API by understanding the response and sending requests to the server - there is no external documentation needed to discover the API.

\paragraph{Self documentation}
Hydra documents itself by providing meta data along the payload. \\
It can use the same data model for generating built-in documentation and serving the API. These documentations are less likely to become out-of-date which reduces maintenance costs.

\paragraph{Collections, partial views and pagination}
In the world of Hydra, there are conceptually only \textit{things} and \textit{lists of things}. There is no list of list of things. \\
Hydra calls a list of things \textit{collection}, which can have things as members. There is the concept of a partial view. This gives the client the ability to fetch a subset of a collection. That partial view can be controlled by client initiated \gls{pagination}. The client provides a limit and an offset when requesting a collection in order to initiate \gls{pagination}.

\paragraph{Operations}
In order to interact with the server, a client has to know which operations the server supports. Hydra exposes all information necessary to build and invoke operations. This allows any client that understands Hydra to interact with it without any prior knowledge about the concrete API implementation. \\
This part is not fully specified yet at the time of writing. \footnote{We had to directly communicate with Hydra core members and reverse engineer tools in the Hydra ecosystem in order to understand how Hydra operations are supposed to work.}

\subsection{Analysis of design languages}
\textit{``A design language or design vocabulary is an overarching scheme or style that guides the design of a complement of products or architectural settings. Designers wishing to give their suite of products a unique but consistent look and feel define a design language for it, which can describe choices for design aspects such as materials, colour schemes, shapes, patterns, textures, or layouts. They then follow the scheme in the design of each object in the suite.''} \citep{designlanguage}

The benefit of using design languages is the consistent look and feel of UIs that are composed of a set of default components. Additionally, it frees UI developers from cumbersome work such as making sure that the UI is responsive and behaves well on various screen sizes. UI developers can communicate and think in a higher level of abstraction by omitting implementation details that are taken care of by the design language implementation. As an example, it allows for the mental model of small buttons horizontally next to each other while ignoring the implementation using CSS properties.

The major design languages are built on top of years of research in human-computer interaction. Evaluation of the theoretical foundation is not in the scope of this thesis. We aim to benefit off existing implementations and tools.

\subsubsection{Design language implementations}
The UI framework will consume React components provided by UI developers and it will use React for internal components as well. The only requirement for the design language implementation is the use of React.

\paragraph{Material Design}
Material Design was developed at Google in 2014. Initially it was used in Google's own mobile apps and in Android \footnote{Android is a mobile operating system developed by Google.}. In order to offer a consistent \gls{ux} across all products, Google uses it in web products as well.

The React implementation of Material Design is called Material UI \footnote{\url{https://material-ui.com/}}. It bottles up the \gls{ux} best practices and provides them in the form of React components. \\
By implementing a simple form with three text fields and a button, we conclude that Material UI is too heavyweight. It is not always obvious how to configure components - especially big ones. The main contribution of Material Design and Material UI are best practices in mobile \gls{ux}. This reaches beyond simple forms and visualization of lists. Material Design dictates the look and feel of the \gls{ux} holistically. \\
The UI framework requires a lower level design language implementation that doesn't make assumptions and doesn't state rules about larger components and \gls{ux}. These decisions have to be made by UI developers using the framework.

\paragraph{Semantic UI}
\textit{``Semantic empowers designers and developers by creating a shared vocabulary for UI.''} \citep{semanticui} This thesis explores the benefits of shared vocabularies by using linked data in HTTP APIs - the shared vocabulary of Semantic UI could be mapped to the shared vocabulary in the data. Semantic UI is not as widely adopted as Material Design. Its React implementation has a level of abstraction that fits generically rendering forms and navigation.

\begin{figure}[!htb]
  \includegraphics[width=150pt]
    {images/materialdesign.png}
  \includegraphics[width=320pt]
    {images/semanticui.jpg}
  \caption{Material Design on a mobile device (left) and Semantic UI components (right). \\ (Source: https://wikipedia.org, https://www.sketchappsources.com/)}
\end{figure}

We pick Semantic UI and its React implementation.
