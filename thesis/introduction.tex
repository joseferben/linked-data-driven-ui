\section{Introduction}\label{introduction}

\begin{quotation}
\textit{``There is no single development, in either technology or management technique, which by itself promises even one order-of-magnitude improvement within a decade in productivity, reliability, in simplicity.''} \citep[p.~1]{nosilverbullet}
\end{quotation}

Frederick P. Brooks, Jr. states this in his work \textit{No Silver Bullet} where he looks for a silver bullet to \textit{lay the werewolves of software complexity to rest} \citep[p.~1]{nosilverbullet}.

He breaks down software complexity and describes that reduction of software complexity is more feasible for some types of complexity than others.

This thesis explores the generation of user interfaces (UIs) for application programming interfaces (APIs) that fulfill certain criteria in the context of the World Wide Web. The main goal is to reduce the manual work required in UI development. The goal of this thesis is to decrease costs of UI development by reducing the involved complexity.

\subsection{Context}\label{context}
In order for information systems to be useful, they either have to interact with other systems or with human users. Those systems are increasingly more connected together and they exchange data without any human interaction. In the context of contemporary UI and web development, UIs are implemented that consume HTTP APIs which use the JSON data format \citep{jsonformat}. Those UIs are part of mobile, desktop or browser applications. End users interact with applications through UIs to carry out tasks.

\subsection{Problem}\label{problem}
JSON is a popular format for data exchange between information systems in web development. Humans are able to infer the meaning of a piece of JSON data but machines are not. Specific knowledge about the JSON data is needed for a machine to understand the meaning of it. That specific knowledge is only valid for a certain API in a certain context. It is manually, often implicitly, encoded into the application containing the UI as shown in figure \ref{fig:hardcoded}. This knowledge is required to understand the data coming from such an API, it gives the data semantics. These client applications are strongly coupled to specific APIs which hinders their re-use. Instead, client applications are written from scratch to work with slightly different APIs, often sharing the same or similar data models.

Current approaches fail to exploit the architecture of the World Wide Web and they completely ignore the possibilities offered by \gls{linkeddata}.

\begin{figure}[!htb]
  \center{\includegraphics[width=450pt]
    {images/ui-dev-now.png}}
  \caption{A part of the business logic is hard coded into the client.}
  \label{fig:hardcoded}
\end{figure}

\subsection{Scope and expected results}\label{sec:scope}
The main question that this thesis aims to answer is: \textit{Is it possible to reduce development costs in UI development by using \gls{linkeddata}?}

The scope and expected results of this thesis include following artifacts:

\begin{itemize}
\item a \gls{hypermedia} vocabulary or a \gls{hateoas} specification
\item implementations of two HTTP APIs using that \gls{hateoas} specification
\item a brief evaluation of design languages and their implementations
\item a proof of concept for the UI framework
\end{itemize}

The choice of the \gls{hateoas} specification is driven by the implementation of the HTTP APIs. The implementation of the HTTP APIs is driven by use cases that resemble real world scenarios. A UI framework is implemented that is used to create a client that can consume HTTP APIs of both use cases. The framework uses a design language implementation internally to provide basic building blocks to the UI developer.

\subsection{Industrial partner}
The industrial partner of this thesis is M. Nieminen Consulting. Miko Nieminen Consulting is interested in pushing open source efforts in \gls{linkeddata} driven UI with complex user interaction. A future real-world use case is the generation of web based back office applications for data and process management.
